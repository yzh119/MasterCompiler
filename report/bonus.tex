\documentclass[12pt]{article}
\usepackage{amsmath}
\usepackage{xeCJK}
\usepackage{amsbsy}
\usepackage{amsthm}
\usepackage{listings}
\usepackage{amssymb}
\usepackage{amsfonts}
\usepackage[colorlinks,linkcolor=red]{hyperref}
\usepackage{graphicx}
\usepackage[top=1in, bottom=1in, left=1in, right=1in]{geometry}
\author{Zihao Ye}
\title{\textbf{Bonus}}
\newtheorem{thm}{Theorem}
\newfontfamily\cons{Consolas}
\newcommand{\ud}{\mathrm{d}}
\newcounter{exer}
\setmainfont{Cambria}
\setCJKmainfont{Microsoft YaHei}
\providecommand{\abs}[1]{\left\lvert#1\right\rvert}
\providecommand{\norm}[1]{\left\lVert#1\right\rVert}
\providecommand{\floor}[1]{\left\lfloor#1\right\rfloor}
\providecommand{\set}[1]{\left\{#1\right\}}

\newenvironment{problem}[1]
    {   \stepcounter{exer}
        \noindent
        \large
        \textbf{Problem \theexer . #1}
        \normalsize
        \smallskip
        
    }
   {}
\begin{document}
\maketitle
\section{成员方法} % (fold)
\label{sec:成员方法}

% section 成员方法 (end)

支持成员函数(方法), 但是没有构造函数/继承/多态。

\subsection{再现方式} % (fold)
\label{sec:再现方式}

% section 再现方式 (end) 

成员函数的例子: example/applyForProfessor.mx

\noindent
期望结果: example/example.out

\subsection{编译方式} % (fold)
\label{sec:编译方式}

% section 编译 (end)

\centering
\begin{lstlisting}[language=bash,
        basicstyle=\small\cons]
make 
cp example/applyForProfessor.mx bin
cp example/example.out bin
cd bin
java -jar Master.jar < applyForProfessor > assem.s
spim -file assem.s > applyForProfessor.out
diff example.out applyForProfessor.out
\end{lstlisting}
\end{document}